\documentclass[tikz,border=1mm]{standalone}

\begin{document}
    
\def\layersep{2.5cm}

\begin{tikzpicture}[shorten >=1pt,->,draw=black!100, node distance=\layersep]
    \tikzstyle{every pin edge}=[<-,shorten <=1pt]
    \tikzstyle{neuron}=[circle,fill=black!25,minimum size=17pt,inner sep=0pt]
    \tikzstyle{input neuron}=[neuron, fill=white!100,draw=black];
    \tikzstyle{output neuron}=[neuron, fill=white!100,draw=black];
    \tikzstyle{hidden neuron}=[neuron, fill=white!100,draw=black];
    \tikzstyle{annot} = [text width=4em, text centered]

% Draw the input layer nodes
\foreach \name / \y in {1,...,3}
% This is the same as writing \foreach \name / \y in {1/1,2/2,3/3,4/4}
    \node[input neuron, pin=left:Input \y] (I-\name) at (0,-\y) {};

% Draw the hidden layer nodes
\foreach \name / \y in {1,...,4}
    \path[yshift=0.5cm]
        node[hidden neuron] (H-\name) at (\layersep,-\y cm) {};

% Draw the output layer node
\node[output neuron,pin={[pin edge={->}]right:Output 1}, right of=H-2] (O1) {};
\node[output neuron,pin={[pin edge={->}]right:Output 2}, right of=H-3] (O2) {};

% Connect every node in the input layer with every node in the
% hidden layer.
\foreach \source in {1,...,3}
    \foreach \dest in {1,...,4}
        \path (I-\source) edge (H-\dest);

% Connect every node in the hidden layer with the output layer
\foreach \source in {1,...,4}
    \path (H-\source) edge (O1);
\foreach \source in {1,...,4}
    \path (H-\source) edge (O2);

% Annotate the layers
\node[annot,above of=H-1, node distance=1cm] (hl) {Hidden layer};
\node[annot,left of=hl] {Input layer};
\node[annot,right of=hl] {Output layer};

\end{tikzpicture}
\end{document}